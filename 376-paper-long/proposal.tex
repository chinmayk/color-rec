\documentclass{article}
\input ../Resources/latex/std-macros
\usepackage{amsmath}\title{CS228: Problem Set 3}

\title{Data driven web site design}
\author{Chinmay Kulkarni}
\date{\today}
\begin{document}
\maketitle
\begin{abstract}
Design processes have traditionally focused on coming up with one right design as their objective. Increasingly, however, it is being understood that there is no one design that is right for all situations and for all users. The cost of creating design variations has also significantly reduced, especially in the domain of web design. 

We argue that using techniques from machine learning, it is possible to revise the design process to create not {\em the right design} but rather designs that are right for the particular user and context. 
\end{abstract}

\Section{one-fits-all}{The futility of one design fits all}
\Section{related-work}{Related Work}
\Subsection{personalization}{Personalization}
\Subsection{ab-testing}{A/B or Multivariate testing}
\Subsection{algorithms}{Learning Algorithms}
\Section{changing-design}{Changing the design process}

\end{document}