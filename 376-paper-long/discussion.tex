In our palette likability study, the \system-generated frequency and distance palettes performed significantly better than the Protovis palettes, which are composed of colors that are aesthetically pleasing. This finding suggests that using topic-relevant colors in a visualization improves the aesthetics of the visualization as a whole. 

At the end of both the color relevance and palette likability surveys, participants filled out an optional feedback form. In our individual color relevance tasks, some participants revealed having trouble picking the best color when they felt more than one color was appropriate for the given topic. According to one participant, ``I thought this was HARD! Some of the time I wanted to pick more than one color (for CAL there was a gold and a blue option). It definitely forced me to think in a way I don't normally think." Our test for \textbf{H1} covers this case, as participants only need to choose the most relevant color, and either color would suffice. Overall, the feedback for the individual color relevance task was quite positive. One participant shared, ``This was a very interesting task and it really made me think about which colors are associated with certain items."

However, the feedback for the palette likability task was much more negative. One participant on this task shared, ``It was sometimes difficult to decide because usually none of the color palettes were colors I would have chosen", while another stated more boldly, ``Most of the color combinations were strange and unappealing." The large discrepancy in tone between the color relevance feedback and the palette likability feedback suggests that while \system is able to generate a set of appropriate colors for a topic, it fails to select the right color to make aesthetically pleasing palettes. 

This result is not altogether surprising. Several topics have multiple colors associated with them, and our assumption that appropriate colors can be extracted from images is only an approximation. The results and participant feedback from these studies inform our design of a mixed initiative strategy in which the computer offers a set of potential palettes, shown in the context of the visualization, and the human is easily able to edit the palette in context by choosing other \system-generated colors, or selecting their another color from a color-picker. 