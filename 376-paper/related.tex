CEK
400 words. 
One para each: Color, mixed initiative, emotions and color (  http://socrates.berkeley.edu/~plab/).

Prior work exists on automatic creation of color palettes. This work falls broadly in two categories. The first focuses on finding representative  colors from images, that can be used as color palettes. The most recent of these is~\cite{morse2007image}. This line of research has so far focused only on extracting colors from a single image. This project extends this work by extracting colors from multiple, related images. I believe that some of the techniques used by~\cite{morse2007image}, such as a weighted histogram that uses color saturation and neighborhood color coherence, can be adapted for multiple images too. Depending on constraints of time, I plan to explore some of these techniques.
	
The second category of research on palette generation focuses on optimizing visual properties, such as color saliency and perceptive color distance, both manual or rule-based, as pioneered by Brewer~\cite{brewer1999color}; and  with varying degrees of automation ~\cite{healey1996choosing, zeileis2009RGBland}. I believe most such optimization research is complementary to this project, and can be used as a post-extraction step to optimize the colors chosen. Statistical work on color saliency is valuable, even if it hasn't been directly applied as a optimization objective; color saliency in the context outside data-visualization in~\cite{chuang2008probabilistic, benavente2002statistical}. 

Topic models have been shown to be effective in information retrieval. Latent semantic analysis (and later, LDA), for instance, has been used to find ``latent'' similarities between concepts~\cite{dumais1988using, blei2003latent}. Similar similarity-measures have been computed for nodes in a graph~\cite{jeh2002simrank}. While these similarity measures may help to better cluster color-values, they don't target the domain of color recommendations directly. 
