Prior work  on automatic palette generation falls broadly in two categories. The first focuses on finding representative  colors from images that can be used as color palettes. Research has so far focused on extracting colors from a single image~\cite{morse2007image}. \system extends this work by extracting colors from multiple related images. Other work optimizes visual properties such as color saliency and perceptual color distance. Optimization can be manual/rule-based~\cite{brewer1999color}; or partially automated~\cite{healey1996choosing, zeileis2009RGBland}. Such optimization research is complementary to \system, and can be used as a post-extraction step to optimize the colors chosen. Other statistical work on color saliency valuable to \system is~\cite{chuang2008probabilistic}. 

Mixed initiative systems have been shown to be useful in a broad variety of applications~\cite{hearst1999mixed}, especially  where AI is an important system component~\cite{healey2008visual}. \system uses the mixed initiative paradigm, and earlier work  informs it of prevalent best practices.
%Topic models have been shown to be effective in information retrieval. Latent semantic analysis (and later, LDA), for instance, has been used to find ``latent'' similarities between concepts~\cite{dumais1988using, blei2003latent}. Similar similarity-measures have been computed for nodes in a graph~\cite{jeh2002simrank}. While these similarity measures may help to better cluster color-values, they don't target the domain of color recommendations directly. 
