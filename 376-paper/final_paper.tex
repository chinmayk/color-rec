\documentclass{article}
\input ../Resources/latex/std-macros
\usepackage{times}
\usepackage{uist}
\usepackage{amsmath}
\usepackage{graphicx}

\begin{document}

% --- Copyright notice ---
\conferenceinfo{UIST'09}{October 4-7, 2009, Victoria, British Columbia, Canada}
\CopyrightYear{2009}
\crdata{978-1-60558-745-5/09/10}

% Uncomment the following line to hide the copyright notice
 \toappear{}
% ------------------------


\title{Colorific: A mixed initiative model for choosing the right color.}

%%
%% Note on formatting authors at different institutions, as shown below:
%% Change width arg (currently 7cm) to parbox commands as needed to
%% accommodate widest lines, taking care not to overflow the 17.8cm line width.
%% Add or delete parboxes for additional authors at different institutions. 
%% If additional authors won't fit in one row, you can add a "\\"  at the
%% end of a parbox's closing "}" to have the next parbox start a new row.
%% Be sure NOT to put any blank lines between parbox commands!
%%

\author{
\parbox[t]{9cm}{\centering
	     {\em Chinmay Kulkarni}\\
	     Stanford University HCI Group\\
              Computer Science Department\\
	     Stanford, CA 94305\\
	     chinmay@cs.stanford.edu}
\parbox[t]{9cm}{\centering
	     {\em Julie Fortuna}\\
	     Stanford University HCI Group\\
              Computer Science Department\\
	     Stanford, CA 94305\\
	     jfortuna@stanford.edu}
}

\maketitle

\abstract
Is it possible to automatically create a color palette relevant to a topic? Could such a palette be used to guide color choices while visualizing data? We envision a tool that automatically creates aesthetically pleasing and topic-relevant palettes for a  large class of topics. In order to do this, we must first extract palettes from color pixel values of images from Google Images via clustering and topic models. 

\classification{H5.2 [Information interfaces and presentation]:
User Interfaces. - Graphical user interfaces.}

\terms{Design, Human Factors, Experimentation}

\keywords{Information visualization, colors, crowdsourcing, user study, mixed initiative}

\tolerance=400 
  % makes some lines with lots of white space, but 	
  % tends to prevent words from sticking out in the margin

\section{INTRODUCTION}
\input introduction.tex

\section{RELATED WORK}
\input related.tex
\section{SYSTEM DESCRIPTION}
\input algorithm.tex
\section{SYSTEM EVALUATION}
\input evaluation.tex
\section{WALKTHROUGH}
JMF An envisioned use case of mixed initiative. Include screenshot. (150 words) \\
David is making a bar chart of the number of games won by a selection of Pac-10 football teams. He wants to color each bar a representative color for the team. However, both the Cal Bears and the UCLA Bruins have the colors blue and gold. Which should be blue, and which should be gold? He loads Colorific to find out. David enters the specific items he is plotting, as well as their values. Colorific plots his values on a bar chart and automatically generates a selection of four potential palettes he can choose from. David selects one generated palette, which looks almost the way he wants it. However, after seeing the results in context, he thinks the Cal Bears bar should be gold instead of blue. He clicks on the bar and is presented with a selection of other Colorific generated colors for the Cal Bears, as well as a color picker that allows him to tweak Colorific colors or select his own. \\
\scalebox{0.5}{\includegraphics{colorific_screenshot.png}}
\section{DISCUSSION}
\input discussion.tex

\section{CONCLUSION}
CEK Make up stuff

\section{ACKNOWLEDGMENTS} 
CEK Thank Jeff, Scott and Jesse.
(BOTH together 150 words)


\bibliographystyle{abbrv}

\bibliography{../Resources/references/color-refs,../Resources/references/recommender-refs}
\end{document}
