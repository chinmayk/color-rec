\documentclass{article}
\input ../Resources/latex/std-macros
\usepackage{times}
\usepackage{uist}
\usepackage{amsmath}

\begin{document}

% --- Copyright notice ---
\conferenceinfo{UIST'09}{October 4-7, 2009, Victoria, British Columbia, Canada}
\CopyrightYear{2009}
\crdata{978-1-60558-745-5/09/10}

% Uncomment the following line to hide the copyright notice
 \toappear{}
% ------------------------


\title{Colorific: A mixed initiative model for choosing the right color.}

%%
%% Note on formatting authors at different institutions, as shown below:
%% Change width arg (currently 7cm) to parbox commands as needed to
%% accommodate widest lines, taking care not to overflow the 17.8cm line width.
%% Add or delete parboxes for additional authors at different institutions. 
%% If additional authors won't fit in one row, you can add a "\\"  at the
%% end of a parbox's closing "}" to have the next parbox start a new row.
%% Be sure NOT to put any blank lines between parbox commands!
%%

\author{
\parbox[t]{9cm}{\centering
	     {\em Chinmay Kulkarni}\\
	     Stanford University HCI Group\\
              Computer Science Department\\
	     Stanford, CA 94305\\
	     chinmay@cs.stanford.edu}
\parbox[t]{9cm}{\centering
	     {\em Julie Fortuna}\\
	     Stanford University HCI Group\\
              Computer Science Department\\
	     Stanford, CA 94305\\
	     jfortuna@stanford.edu}
}

\maketitle

\abstract
Is it possible to automatically create a color palette relevant to a topic? Could such a palette be used to guide color choices while visualizing data? We envision a tool that automatically creates aesthetically pleasing and topic-relevant palettes for a  large class of topics. In order to do this, we must first extract palettes from color pixel values of images from Google Images via clustering and topic models. 

\classification{H5.2 [Information interfaces and presentation]:
User Interfaces. - Graphical user interfaces.}

\terms{Design, Human Factors, Experimentation}

\keywords{Information visualization, colors, crowdsourcing, user study}

\tolerance=400 
  % makes some lines with lots of white space, but 	
  % tends to prevent words from sticking out in the margin

\section{INTRODUCTION}
TODO Julie/Chinmay. 
\section{RELATED WORK}
Prior work exists on automatic creation of color palettes. This work falls broadly in two categories. The first focuses on finding representative  colors from images, that can be used as color palettes. The most recent of these is~\cite{morse2007image}. This line of research has so far focused only on extracting colors from a single image. This project extends this work by extracting colors from multiple, related images. I believe that some of the techniques used by~\cite{morse2007image}, such as a weighted histogram that uses color saturation and neighborhood color coherence, can be adapted for multiple images too. Depending on constraints of time, I plan to explore some of these techniques.
	
The second category of research on palette generation focuses on optimizing visual properties, such as color saliency and perceptive color distance, both manual or rule-based, as pioneered by Brewer~\cite{brewer1999color}; and  with varying degrees of automation ~\cite{healey1996choosing, zeileis2009RGBland}. I believe most such optimization research is complementary to this project, and can be used as a post-extraction step to optimize the colors chosen. Statistical work on color saliency is valuable, even if it hasn't been directly applied as a optimization objective; color saliency in the context outside data-visualization in~\cite{chuang2008probabilistic, benavente2002statistical}. 

Topic models have been shown to be effective in information retrieval. Latent semantic analysis (and later, LDA), for instance, has been used to find ``latent'' similarities between concepts~\cite{dumais1988using, blei2003latent}. Similar similarity-measures have been computed for nodes in a graph~\cite{jeh2002simrank}. While these similarity measures may help to better cluster color-values, they don't target the domain of color recommendations directly. 

\section{SYSTEM DESCRIPTION}
\input algorithm.tex
\section{SYSTEM EVALUATION}
\input evaluation.tex

\section{RESULTS}
blah blah

\section{DISCUSSION}
blah



\bibliographystyle{abbrv}

\bibliography{../Resources/references/color-refs,../Resources/references/recommender-refs}
\end{document}
