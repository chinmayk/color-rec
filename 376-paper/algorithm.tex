550 Words

The overall goal of \system is to find colors that are relevant to a given topic. To do this, \system assumes that images related to a topic contain colors that are relevant to the topic. In particular, it assumes that there is a mapping, say $\sC(t)$, from the set of images for a topic $t$, $I(t)$, and the set of colors, $C(t)$ that are relevant to it. 
\begin{align*}
\sC(t) : I(t) &\rightarrow C(t)
\end{align*} 

Therefore, the main contribution of \system is to find an efficient way to compute $\sC(t)$. 
We use Google Images as a means to find images related to a given topic. 

\system then randomly samples a number of pixels from such images. Given our assumption that images are a good source of color for topics, random sampling from a set of images is used as a way to randomly sample from the topic-space.   

\Subsection{query-system}{Query System}
Google Images is queried for images from  the category.
\Subsection{statistical}{Statistical summarization}
We assume that the images from the category are a random sampling from the concept-space of the category. Taking this assumption further, we look at the {\em average} frequencies of the different colors as a metric of how concepts are shared across the values in a category.

\begin{align}
Old = \alpha*average + (1-\alpha)*new \\
  new = \frac{(old - \alpha*average)}{(1-\alpha)} 
\end{align}

Since we are interested in the colors specific to a category value, we substract a fraction of the average color frequency. 

\Subsection{clustering}{Clustering}
We cluster the result to get relevant colors in LAB space. We found that low saturation colors are less likely to be relevant, 
so we reweight more saturated colors to be more relevant.
